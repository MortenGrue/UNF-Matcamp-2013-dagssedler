\begin{minipage}[t]{100mm}
\vspace{1mm}
\section*{Produktandmeldelse -- Printer A2}
Ved første øjekast ses straks at der er kræset for detaljen, printeren som har været ansvarlig for at printe alle det fantastiske faglige materiale, selv denne sublime anmeldes af A2 printeren er printet på A2 printeren.  Der findes på IMF også A3 og A4 printer, og på navnene skulle man tro de var bedre, men ikke på IMF, her er A2 konge. Kun A2 kan printe A2 papir, kun den kan valse, hæfte og bøje papir! Med sine 90 ”single side” per minut er den, den rene gatling gun, totalt urørt som den absolutte hersker. De ny tider har bragt 3D-Printere, og i lyset af denne banebrydende teknologi skal A2-Printeren også dømmes. Denne ekstra dimension udspænder sig dog kun i det euklidiske rum, og er altså kun anvendelig i denne ene metrik, denne åbenlyse begrænsning af den tredje dimension gør at 3D printere er total ubrugelige for rigtige matematikere.  A2 printerens blanke matte design og grålige nuance giver en uimodståelig seksuel tiltrækning, som er så stærk at redaktionen på UNF Newz har, måtte indføre regler for at undgår usømmelig omgang med A2. 
A2 Printeren indeholde mange små dele og bør derfor ikke gives til børn under 3 år.

\vspace{1mm}
\section*{Madandmeldelse -- DSB-kaffe}
\fcolorbox{black}{white}{$10^{-6}$ af $\aleph_0$ chokoladekiks}
\vspace{2mm}

Jeg havde glædet mig til at skifte til IC-toget hele vejen siden at være gået i land efter en svømmetur i Smålandsfarvandet og lige havde nået bumletoget fra Bandholm station. Efter flere timers togtur mellem kålmarker, ville jeg nu endelig få dagens første kaffe. Efter uden succes at have lyttet til højtaleren i toget, gik jeg ud at søge efter salgsvognen. Det viste sig at denne i dagens anledning kun tog imod svenske $50$ øre mønter, som dog kunne fremskaffes ved en kort opringning til Gefion, som dog havde tømmermænd og derfor ikke tog det særlig pænt. Efter at have betalt mere end en husalfs årsløn (plus moms) for kaffeen, blev den serveret i et papbæger, varmt nok til at man brændte alle finge (og en tå) inden det første tår kunne tages. Derimod viste kaffen sig at være koldt, så snart den ankom i munden, mindende om noget man finder i kotorets kaffemaskine tidligt mandag morgen efter en forlænget weekend. Ordet kaffe, var dog heller ikke passende, da væsken nok aldrig havde mødt noget der mindede om kaffebønder, nok mindre end de fleste kaffeerstatninger fra krigens tid, heriblandt vand tilsat lidt muld. Jeg drak dog hele koppen, ofrende mit velvære for avisens kvalitet og overlevede dog, hjulpet på vej af i forvejen manglende livslyst. Vågen blev jeg dog, måske mere som en overlevelsesreaktion end som følge af kaffens koffein.


\end{minipage}

