\begin{minipage}[t]{100mm}
\vspace{3mm}
\section*{Løsning til gårdsdagens opgave}
\subsection*{Løsning til 1. hatteopgave}
Det er tilstrækkeligt med $k$ døre, hvor $k$ er antallet af farver. Dværgene tildeler på forhånd værdierne $1,\dotsc,k$ til de forskellige mulige farver. Antag at dværg $i$ får en hjelm af farve $a_i$ på hovedet. Enhver af dværgene summerer så værdierne for alle hjelmene han ser, og han regner cyklisk omkring $k$, sådan at $k=0$, $k+1=1$, $k+2=2$, osv.. Derefter går dværgene gennem døren med samme tal som den sum de hver især regnede sig frem til.

Lad $S$ være den samlede sum for alle hjelme, $S=a_1+a_2+\dotsb+a_n$. Hjelmene som dværg nummer $i$ kan se, har så summen $S-a_i$ fordi han ikke kan se sin egen hjelm. Dværgen $i$ går derfor gennem dør $S-a_i$. To dværge $i$ og $j$ går derfor gennem samme dør hvis og kun hvis $S-a_i=S-a_j$. Dette er tilfældet netop når $a_i=a_j$, dvs. hvis dværg $i$ og $j$ har samme farve hjelm.

\subsection*{Løsning til 2. hatteopgave}
Lad $M$ være mængden af dværge i hallen, og lad $F$ være mængden af mulige farver. En fordeling af farvede hjelme er nu det samme som en funktion $f\colon M\to F$ der for enhver dværg $m\in M$ fortæller hvilken farve $f(m)$ som $m$'s hjelm har. Vi indfører nu en ækvivalensrelation mellem de mulige fordelinger: To fordelinger $f_1\colon M\to F$ og $f_2\colon M\to F$ er ækvivalente, skrevet $f_1\sim f_2$ hvis der kun er endeligt mange $m\in M$ hvor fordelingerne afviger $f_1(m)\neq f_2(m)$.

En fordeling $f$ afviger ikke fra sig selv, så $f\sim f$ gælder altid. Per symmetri gælder $f\sim g$ hvis og kun hvis $g\sim f$. Hvis $f$ kun afviger endeligt mange steder fra $g$, og $g$ kun afviger endeligt mange steder fra $h$, så vil $f$ kun afvige endeligt mange steder fra $h$. Vi har altså som påstået en ækvivalensrelation.

Dværgene vælger nu på forhånd en specifik fordeling fra hver ækvivalensklasse. Da der er uendeligt mange dværge, kræver dette valg at dværgene bruger udvalgsaksiomet.

Herefter når dværgene får hjelme på -- kald den faktiske fordeling for $f$. Enhver dværg $m$ vil så være i stand til at se alle farver $f(m')$ for alle $m'\neq m$. Dværgene vil altså kende den faktiske fordeling bortset fra sin egen hjelm, og da dette er en endelig fejl, vil dværgene være i stand til at bestemme ækvivalens $f$. Lad $f_0$ være den entydige fordeling som dværgene valgte på forhånd fra ækvivalensklassen med $f$. Enhver dværg $m\in M$ gætter så på farven $f_0(m)$, så dværgenes samlede gæt er fordelingen $f_0$. Idet vi har $f_0\sim f$, afviger den gættede fordeling $f_0$ kun endeligt mange steder fra den virkelige fordeling $f$.

\begin{center}
\section*{Dagens opgave}
Why is a raven like a writing desk?
\end{center}

\end{minipage}
