\begin{minipage}[t]{100mm}
\vspace{1mm}
\section*{Campens forfatningsretlige status}
Det har i årevis været et omdiskuteret emne blandt lærde i sommerskoleforfatningsvidenskab, om campen som organisation skal anses som \emph{corporation sole} eller som \emph{corporation aggregate}. I lighed med forholdene, der gælder for den britiske krone, er førstnævnte den mest udbredte interpretation, og tydeligvis også den, der ligger til grund for ovenstående resolution. Samtidig bruger resolutionen dog campens flerpersonelle egenskab for at nedsætte et konsitorium, som overtager nogle af absolutens beføjelser i en begrænset periode. Dette minder om et typisk første skridt i vestlige monarkiers demokratiseringsproces. Da denne afståelse dog er begrænset til få timer, må vi nærmere anse resolutionen som et middel til at fatsholde den absolutte magt, på trods af en stærk opposition i arrangørkredsen, fremfor at anse dette for det første spæde skridt i en demokratiseringsproces. Absoluten er desuden kendt for tidligere at have afgivet magten til yngre kræfter, udelukkende for at erobre den tilbage med magt i efterfølgende år, og sende de demokratiske kræfter til strafophold i kolonierne, sammen med deres uskyldige, men ofte forræderiske familie. Sammenfattende må vi derfor observere, at campen forbliver monarkist, og dermed til stadighed er i et modsætningsforhold til de teknokratiske og oligarkiske forfatningsstrukturer vi observer på andre camps, dog primært i oversøiske territorier. Erfaringerne fra disse reformationsprocesser er dog brogede, og prægede af gennemgribende spritmisbrug.  Dette, og den omstændighed, at jeg er blevet truet med eksklusion og eksekution af Campmarskallatet, indebærer at jeg trygt anbefaler omretholdelsen af de Gaussgivne absolutisk-aksiomatiske magtstrukturer på campen.

{\flushright\emph{S.O.M., campforfatningsekspert}}

\section*{Ninjas dagbog}
Denne dag vil have været bedre end den foregående. Den vil være startet med en inspirerende, men lidt for voldsforladt velkomst hvorefter jeg vil have tilbragt meget af dagen med den præcise udførsel af administrative opgaver af strategisk betydning, som fremadrettet vil kunne betyde sejr eller nederlag i den afsluttende kamp mod sørøverne. Men mange af ånderne vil være blevet frarøvet deres selvstændighed og tvunget i den rette topologi. Sidenhen vil jeg i stor frygt have observeret en ukendt skikkelse, der havde betrådt bygningen. Aftenen vil så have sluttet med, at jeg med afsky vil have håndteret grædende ånder, som ikke modstod deres frygter, og overvejet gang på gang, om jeg burde have frigjort dem fra deres lidelser ved et dræbende sving af mit håndled. Af respekt for de lokale traditioner, som kræver afsyngning af H.C. Andersens \emph{Soldaten} inden alle henrettelser, vil jeg dog have været tvunget til, af mangel på et passende kor, at aflyse drab og begrænse mig til psykisk tortur inden nattesøvnen.

{\flushright\emph{Ninja, savner sine drab}}

\section*{Tegnsætning}
Der, er, blevet, klaget, over, at, der, mangler, kommaer, forskellige, steder, i, vores, skrevne, materialer, så, de, overskydende, kommaer, i, denne, tekst, kan, frit, bruges, i, de, andre, artikler, i, avisen, da, der kun, er, seksten, artikler, men, mere, end, hundredetres, kommaer, må, der, nødvendigvis, være, mindst, en, artikel, der, indeholder, mere, end, ti, kommaer.

\end{minipage}

